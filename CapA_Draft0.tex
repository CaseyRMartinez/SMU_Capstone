\documentclass[runningheads]{llncs}

\usepackage{blindtext}
\usepackage{lipsum}
\usepackage{amsmath}
\usepackage{amsfonts}
\usepackage{amssymb}
\usepackage{float}
\usepackage{graphicx}
\usepackage{booktabs}
\usepackage{array}
\usepackage[colorlinks=true,citecolor=black,linkcolor=black,filecolor=magenta,urlcolor=cyan]{hyperref}

\title{Estimating Wind Turbine Blade Condition with Advanced Machine Learning } 

\author{Casey R. Martinez, Festus A. Yeboah, James S. Herford, Matt Brzezinski} 

\institute{$^1$Master of Science in Data Science \\ Southern Methodist University \\ Dallas, Texas USA \\ 
\email{\{casianom,fasareyeboah,jherford\}@smu.edu}}
  
\begin{document} 

\maketitle 

\begin{abstract} 

In order to transition global energy production to more sustainable sources, improving the performance of wind turbines is critical. Unlike solar energy, capture the kinetic energy of wind requires complex mechanical systems which require sophisticated maintenance and planning strategies. There are many traditional approaches to monitoring the internal gearbox and generator but the condition of turbine blades can be difficult to measure and access. Accurate and cost-effective estimates of turbine blade remaining life will drive optimal investments in repairs and improve overall performance. This will drive down costs and help provide cheap and clean electricity for the planet.

In this thesis, we will use advanced machine learning techniques to model blade remaining life using turbine historical time-series data and blade inspection assessments. These inspections will be used as training labels for a multivariate supervised classification approach. 


%In this paper, we present the basic approach and guidelines for writing technical papers for the {\em SMU Data Science Review Journal\/} ({\em DSRJ\/}).  The DSRJ is an open access peer reviewed journal targeted towards all aspects of research and practice, including case studies, technologies, systems, theory, and methodologies, related to the broad field of data science.  Both work in progress and completed research papers are appropriate for the DSRJ. The DSRJ reaches an audience far wider than Southern Methodist University (SMU) and also wider than the data science community.  The quality and presentation of DSRJ content reflects upon the authors and upon SMU.  This paper is intended to provide guidance on the content and structure of a basic article submitted to the DSRJ.  The format used for all DSRJ submissions is specified elsewhere. By following the guidance in this paper, authors should be able to create a well written, well organized, and easily read article that is accessible to the broad audience that reads the DSRJ. 

\end{abstract} 

  

%\begin{keywords} 

%Data Science, Predictive Analytics, Big Data, Data Analytics, Data Engineering, Business Intelligence 

%\end{keywords} 

  

\section{Introduction} 

\subsection{Motivation}   

Modern-day wind turbines are complex and sophisticated machines with hundreds of moving parts. A team of technicians and engineers are needed to maintain operation at levels needed to provide cheap electricity to market. Most of the rotating equipment inside the nacelle portion of the turbine have temperature and vibration sensors to alert the control system of anomalous equipment behavior. Since the fiberglass blades themselves have no moving parts and can be 50 meters and longer, similar monitoring systems have not been implemented at scale. The industry relies on periodic inspection via telescoping photography, drone photography, or human inspection with ropes and harnesses. All of these approaches require turbine downtime and can be labor intensive to collect the data and make blade assessments. 

Since it is not cost effective to inspect every turbine in the fleet, only about half of turbines are inspected and they can go years without follow up inspections. These inspections can also be prone to human error due to the subjective nature of the assessments which are usually based on photography. A more statistical approach would be greatly desired.

\subsection{Problem Statement} 

It has been shown that blade wear is not consistent across wind farms or even across turbines at the same farm. It is believed that certain environmental conditions contribute to this wear such as rain, ice, wind sheer, and wind turbulence. How can we use weather data collected at each turbine along with existing blade inspection data to accurately predict the blade wear for turbines that have not had an inspection? Can we create a model that can, not only estimate the current condition of the blade, but also its rate of wear. Since blade wear contributes to turbine under-performance, can this lost of performance also be estimated. If so, blade repair projects can be more accurately justified. 


%Writing is an art form where the authors tell a story that is intended to convey something to the reader.  In works of fiction, the story may be used to convey a life lesson, to tell of an imagined adventure, or simply to entertain and evoke an emotion.  In works of non-fiction, the story typically seeks to tell a truth or technique or some other aspect of the world.  For the {\em SMU Data Science Review Journal\/} ({\em DSRJ\/}), your non-fiction story should tell a very direct, well organized,  engaging, and compelling story of your problem, solution, results, and conclusions.  
%
%The goal of a technical paper is to change people's thoughts, opinions, and behaviors related to a specific problem.  This means first and foremost that you must tell your story in such a way that the reader is able to understand the problem you addressed, understand the details of how you solved the problem, and understand the importance of both the problem and your results.  If your paper is unclear on any of these points, your paper will not be compelling and will not achieve its goal. 
%
%A compelling paper at the very least convinces the audience that the problem addressing is interesting, that the problem is hard to solve,  that you solved the problem, and that you have made novel contributions  

%-------------------------from syllabus
%Draft Zero is expected to be at least one page in length, including the draft title, author list (include all advisors as authors in the list), abstract draft, and, within the Introduction section, one paragraph motivation and one paragraph problem statement. Teams are expected to have performed a related work and tutorial topic search. This information should be included in Draft Zero in some form. A preliminary set of sections with their titles should be created to indicate the general and expected flow of the paper. This is particularly important for you to identify the tutorial sections required to ensure that the general audience reader will be able to understand your work.

%  
%
%\section{Broad Paper Themes} 
%
%  
%
%\subsection{Stay on Message} 
%
%  
%
%The DSRJ audience consists primarily of people that are either trained in data science  or a related field or work with data in some way. In general, this means that you, the author, know more about your particular problem, the particular application domain in which you are solving your problem, and your solution approach than most of your readers. 
%
%  
%
%\subsection{Define Your Problem} 
%
%  
%
%Defining your the problem you are addressing in your work is the most important step in both coming up with and evaluating a solution and in stating and motivating the problem clearly in your paper. 
%
%  
%
%For ill-defined and open problems, it is very likely that the problem statements made early in the research and in early drafts of your paper will be found to be too broad or general.  As you learn more about a problem it is natural to find and define sub-problems that need to be solved.  And, it is natural to find more specific ways to define your problem based upon this increasing intimacy with the broad problem you are addressing.  
%
%  
%
%\subsection{Motivate Your Problem} 
%
%  
%
%In motivating a problem, your goal is to `set the stage' for the specific problem you are addressing.  This means that your motivation must at least define the broad problem domain and explain to the reader why the reader cares or should care about the broad problem.  Your motivation should also, to the greatest extent possible, motivate the particular problem characteristics that exist in the specific problem that you have defined and solved. 
%
%  
%
%Your motivation frames the problem by placing it in a specific context; thereby, transforming your problem from an abstract idea to a concrete and practical issue that is easily related to by the reader. 
%
%  
%
%\subsection{Paper Organization, Paper Organization, Paper Organization} 
%
%  
%
%Your paper must be organized in a clear and logical fashion. 
%
%  
%
%Communicate the broad ideas and themes of your paper early, and then fill in the details and background information in the body of the paper. 
%
%  
%
%\section{Paper Sections and Contents} 
%
%  
%
%\subsection{Details to Include} 
%
%  
%
%Your paper is telling a story that conveys specific ideas and arguments.  In order for the reader to properly understand your story and be moved by it, you, the author, must write in a way that carries the reader along, step by step, in such a way that your conclusions are their conclusions and that their thoughts, opinions, and behaviors are changed as you intend.  This requires that you include all the details that are relevant to your story and to the reader's understanding of your story including relevant background knowledge required for that understanding.  But, this also requires that you {\em do not\/} include information that does not enhance understanding or move your story along.  The inclusion of not relevant or tangential information, even if it is really interesting, serves only to distract the reader from the goal of your paper. 
%
%  
%
%Do not dwell on implementation details except insofar as the contribute to your story and the main goals of your paper.  You should focus on the algorithms that were implemented not how the algorithms were implemented.   
%
%  
%
%\subsection{The Abstract} 
%
%  
%
%The Abstract is a 200 word executive summary of your paper.  This is the brief elevator pitch that gives a potential reader the very big picture and convinces them to read the paper.  
%
%  
%
%The basic structure of the Abstract follows the general structure of the paper. 
%
%  
%
%\subsection{The Introduction} 
%
%  
%
%The introduction should motivate your problem, state your problem clearly, and communicate the main ideas of your work (such as your techniques for solving the problem, your main results, and your main conclusions) clearly and precisely.  The body of the paper will expand upon the themes of your Introduction in the same order in which you present them in the Introduction.   
%
%  
%
%The purpose of the Introduction section is to provide an executive summary of your paper, and in so doing, provide the holistic overview of all of the important features of your paper. 
%
%  
%
%A reader who understands the big picture of what is being presented in a paper, including the structure, the big ideas, the general techniques, and the main conclusions of the paper, will be able to better understand the details presented therein.  
%
%  
%
%% from IEEE 
%
%A well-written introduction can convince a busy researcher to invest time in reading your article. Craft a better introduction by following these tips: 
%
%Structure your introduction like a funnel, moving from broad to narrow: start by identifying the broad area of a study, continue to a narrower focus on a literature review of your specific research area, and then to an even narrower focus on your specific research question 
%
%Include a comprehensive but concise literature review; provide enough information to establish a context for your work but avoid including studies that are only marginally relevant 
%
%Clearly state your research question and why it's important to the field 
%
%Briefly explain your method for approaching the problem and why you chose that method 
%
%%%%%%% 
%
%  
%
\subsection{Tutorial Background} 
Leading edge blade erosion caused by water droplet impingement reduces blade aerodynamics and limits power output. If left uncorrected, the structural integrity of the blade can be compromised. 
  

%The background sections should cover the foundational material needed to understand the problem and your solution to it.  The sections containing the background material should be appropriately titled. (NOTE: `Background' is NOT an appropriate title.) 
%
%  
%
%\subsection{Solution Approach} 
%
%  
%
%In detailing your solution to the problem, be sure to include pseudocode for any algorithms or methods not covered in the tutorial background sections including your general methodology. Your sections should be appropriately titled.  (NOTE: `Methodology' or any such variation is NOT an appropriate title.) 
%
%  
%
\section{Results} 
%
%  
%
%You should have a results section that compares your obtained results with the results from other methods or approaches and with other people's results as appropriate. 
%
%  
%
\section{Analysis} 
%
%  
%
%Analyze your results.  This analysis may be combined with the results section. 
%
%  
%
%\section{Ethics} 
%
%  
%
%You must have a 1 plus page section on ethics as they relate to your work. The purpose of this section is for you to examine some aspect of ethics as it relates to your problem and your solution. Note that `there are no ethical issues' is NOT an appropriate analysis.  In discussing ethics, be sure to clearly state the ethical issue being discussed or potentially impacted by your work. I'll note that `privacy' is not an ethical issue.  
%
%  
%
%\subsection{Conclusions} 
%
%  
%
%Your Conclusions section (plural since you will have more than one conclusion stated) may summarize your work, but it should contain at least one, preferably three, conclusions drawn from and supported by your analysis and your results.  
%
%  
%
\section{Conclusions} 
%
%  
%
%This is a very brief and incomplete paper covering some of the points that you must consider as you write a paper for the SMU Data Science Review Journal. This source \LaTeX\ file as serves as a basic template that can be used by \LaTeX\ novices as the starting point for their papers. 

\section{Future Work}

\section{References}

\section{Appendices}  

\end{document} 








