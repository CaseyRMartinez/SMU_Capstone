\documentclass[runningheads]{llncs}

\usepackage{blindtext}
\usepackage{lipsum}
\usepackage{amsmath}
\usepackage{amsfonts}
\usepackage{amssymb}
\usepackage{float}
\usepackage{graphicx}
\usepackage{booktabs}
\usepackage{array}
\usepackage[colorlinks=true,citecolor=black,linkcolor=black,filecolor=magenta,urlcolor=cyan]{hyperref}

\title{Estimating Wind Turbine Blade Condition with Advanced Machine Learning } 

\author{Casey R. Martinez, Festus A. Yeboah, James S. Herford, Matt Brzezinski} 

\institute{$^1$Master of Science in Data Science \\ Southern Methodist University \\ Dallas, Texas USA \\ 
\email{\{casianom,fasareyeboah,jherford\}@smu.edu}}
  
\begin{document} 

\maketitle 

\begin{abstract} 

In order to transition global energy production to more sustainable sources, improving the performance of wind turbines is critical. Unlike solar energy, capture the kinetic energy of wind requires complex mechanical systems which require sophisticated maintenance and planning strategies. There are many traditional approaches to monitoring the internal gearbox and generator but the condition of turbine blades can be difficult to measure and access. Accurate and cost-effective estimates of turbine blade remaining life will drive optimal investments in repairs and improve overall performance. This will drive down costs and help provide cheap and clean electricity for the planet.

In this thesis, we will use advanced machine learning techniques to model blade remaining life using turbine historical time-series data and blade inspection assessments. These inspections will be used as training labels for a multivariate supervised classification approach. 


%In this paper, we present the basic approach and guidelines for writing technical papers for the {\em SMU Data Science Review Journal\/} ({\em DSRJ\/}).  The DSRJ is an open access peer reviewed journal targeted towards all aspects of research and practice, including case studies, technologies, systems, theory, and methodologies, related to the broad field of data science.  Both work in progress and completed research papers are appropriate for the DSRJ. The DSRJ reaches an audience far wider than Southern Methodist University (SMU) and also wider than the data science community.  The quality and presentation of DSRJ content reflects upon the authors and upon SMU.  This paper is intended to provide guidance on the content and structure of a basic article submitted to the DSRJ.  The format used for all DSRJ submissions is specified elsewhere. By following the guidance in this paper, authors should be able to create a well written, well organized, and easily read article that is accessible to the broad audience that reads the DSRJ. 

\end{abstract} 

  

%\begin{keywords} 

%Data Science, Predictive Analytics, Big Data, Data Analytics, Data Engineering, Business Intelligence 

%\end{keywords} 

  

\section{Introduction} 

\subsection{Motivation}   

Modern-day wind turbines are complex and sophisticated machines with hundreds of moving parts. A team of technicians and engineers are needed to maintain operation at levels needed to provide cheap electricity to market. Most of the rotating equipment inside the nacelle portion of the turbine have temperature and vibration sensors to alert the control system of anomalous equipment behavior. Since the fiberglass blades themselves have no moving parts and can be 50 meters and longer, similar monitoring systems have not been implemented at scale. The industry relies on periodic inspection via telescoping photography, drone photography, or human inspection with ropes and harnesses. All of these approaches require turbine downtime and can be labor intensive to collect the data and make blade assessments. 

Since it is not cost effective to inspect every turbine in the fleet, only about half of turbines are inspected and they can go years without follow up inspections. These inspections can also be prone to human error due to the subjective nature of the assessments which are usually based on photography. A more statistical approach would be greatly desired.

\subsection{Problem Statement} 

It has been shown that blade wear is not consistent across wind farms or even across turbines at the same farm. It is believed that certain environmental conditions contribute to this wear such as rain, ice, wind sheer, and wind turbulence. How can we use weather data collected at each turbine along with existing blade inspection data to accurately predict the blade wear for turbines that have not had an inspection? Can we create a model that can, not only estimate the current condition of the blade, but also its rate of wear. Since blade wear contributes to turbine under-performance, can this lost of performance also be estimated. If so, blade repair projects can be more accurately justified. 

\subsection{Tutorial Background} 
Leading edge blade erosion caused by water droplet impingement reduces blade aerodynamics and limits power output. If left uncorrected, the structural integrity of the blade can be compromised. 
  

\section{Results} 

\section{Analysis} 

\section{Conclusions} 

\section{Future Work}

\section{References}

\section{Appendices}  

\end{document} 








